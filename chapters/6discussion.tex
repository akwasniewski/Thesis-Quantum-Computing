\section{Discussion}
While the core of this thesis is theoretical, it is worth discussing how quantum computing can be potentially utilized; the state of current advances in physical implementations and their limitations. As of 2025, relatively big, 100 state qubit machines are widely accessible for research centres and quantum-oriented companies worldwide. In theory, those machines should allow for the implementation of discussed algorithms effectively for small input; however, those machines are incredibly noisy, and any prolonged computation is close to impossible as the amount of noise generated renders the final output indistinguishable from noise. \\\\
Various implementations of quantum computers have emerged, each with their own benefits and drawbacks; this includes, among many: trapped ions, superconducting qubits, and photonic qubits, all of which can't overcome the toughest burden - quantum states are incredibly fragile. During computation, qubits are in a superposition of many states, and any disturbance, magnetic field, thermal oscillations, and even cosmic radiation can cause them to lose their quantum state; this problem is known as decoherence. Coherence time, which measures how long a qubit stays in its state before it gets disrupted, on current hardware is on the order of milliseconds. Additionally, quantum gates themselves are not error-free. Many attempts have been made for quantum error correction, none of which were successful enough to allow complex computation.
\\\\
Despite those limitations, small instances of discussed algorithms were successfully implemented. Grover's search has been implemented for small 3-qubit instances \cite{Figgatt2017Grover}. Furthermore, Simon's algorithm, which demonstrates separation between classical and quantum complexity classes, has been demonstrated for 2-qubit instances \cite{Tame2014}. Even factoring was successfully solved; nonetheless, the largest number factored using Shor's algorithm was only 21, back in 2012 \cite{shor21}. Although there have been reports of factoring much larger numbers using different quantum approaches, all those attempts relied heavily on classical factoring, reducing the number of qubits and operations needed drastically; thus, they are widely recognized as just pretending to properly utilize quantum advantage \cite{Smolin_2013}. Those results - although seemingly small in size, and not impressive - demonstrate feasibility, and the prospect that if hardware limitations are overcome, the now rather theoretical algorithms, may prove useful.
\\\\
In the past, most cryptographic algorithms, such as RSA and Diffie Hellman, based their security on factoring hardness. In theory then, Shor's algorithm can solve those problems effectively, which greatly contributed to the staggering amount of attention quantum computing has received in recent years. However, typical key sizes used with these algorithms are a few thousand bits long; therefore, with our current advances in factoring, we are likely decades away from any attempts to solve real-world ciphers using a quantum computer. Besides, as a result of this potential problem, cryptographers started to transition to algorithms that don't rely on integer factoring, so-called post-quantum algorithms. 